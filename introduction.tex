
\section{Введение}

Многие нейродегенеративные заболевания (болезни Альцгеймера, Паркинсона, Хантингтона) сопровождаются накоплением денатурированного белка и образованием нерастворимых агрегатов в нейронах. Эти агрегаты вызывают целый комплекс паталогических процессов в клетке, сходных в различных эукариотических клетках от человека до дрожжей \emph{Saccharomyces cerevisiae}. 

Группа нейродегенеративных заболеваний, называемых синуклеинопатии, ассоциирована с неправильным фолдингом и накоплением α-синуклеина. Одно из них -- болезнь Паркинсона (БП) -- является вторым в мире по распространнености нейродегенеративным заболеванием после болезни Альцгеймера.  Для исследования клеточного противодействия болезни Паркинсона мы использовали дрожжевую модель этой болезни.

α-Синуклеин -- небольшой белок, обнаруженный только у позвоночных. Он широко экспрессирован в головном мозге, локализуется около синаптических окончаний. Физиологическая роль этого белка не ясна. 

α-Синуклеин является основным компонентом телец Леви -- белковых клеточных включений -- главной нейропатологической характеристики БП. Большинство случаев БП идиопатические, их причина не выявлена. Но существуют наследственные, моногенные формы, которые вызываются ду- или трипликацией гена α-синуклеина, некоторыми миссенс мутациями в его кодирующей части.

Пекарские дрожжи \emph{Saccharomyces cerevisiae} благодаря простоте генетических манипуляций и легкости культивирования являются удобной экспериментальным модельным организмом. Дрожжи успешно используются для изучения аспектов болезни Паркинсона, связанных с функционированием и накоплением синуклеина в клетках, который токсичен для дрожжей при оверэкспрессии.

Но, как известно, нейроны -- это дифференцированные клетки, которые не делятся и находятся в стадии G\textsubscript{0} клеточного цикла. Дрожжи же активно делятся. Мы предположили, что неделящиеся, аррестованные в фазе G\textsubscript{0} клетки могли бы быть более реалистичной моделью нейронов. В качестве такой модели был использован штамм cdc53-1, который растет и делится при пермесивной температуре 25°C, но аррестуется в фазе G\textsubscript{0} при 37°C.

Отличаются ли делящиеся и неделящиеся дрожжи по механизму токсичности синуклеина? отличаются ли гены, вовлеченные на эти механизмы? Ответ на эти вопросы может дать генетический скрининг, проведенный в условиях разных моделей.

Общая схема генетического скрининга -- в выбранные модельные штаммы вводится геномная библиотека, из всех трансформированных клеток по критерию выбираются клоны, обладающие необходимыми свойствами. Мы планировали использовать оверэкспрессионную и делеционную геномные библиотеки, в качестве моделей -- делящиеся и неделящиеся дрожжи, индуцибельно экспрессирующие синуклеин, в качестве силы отбора -- токсичность синуклеина для штаммов, не обладающих генетическим преимуществом.

\textbf{Целью} данной работы являлся подбор условий для поиска генов, влияющих на токсичность α-синуклеина в дрожжах \emph{S.~cerevisiae} с термочувствительной мутацией cdc53-1 и дикого типа.

Для выполнения цели были сформулированы следующие \textbf{задачи}:
\begin{enumerate}

\item размножить коллекции плазмид, содержащих мультикопийную и делеционные библиотеки генов \emph{S.~cerevisiae};

\item создать целевые штаммы дикого типа W303 и cdc53-1, экспрессирующих α-синуклеин, и охарактеризовать токсичность α-синуклеина для них;

\item отработать методику эффективной трансформации мультикопийной и делеционных библиотек в штаммы дрожжей cdc53-1 и дикого типа, экспрессирующих α-синуклеин;

\item сравнить и протестировать различные способы скрининга штаммов, экспрессирующих синуклеин, на повышение выживаемости при трансформации геномными библиотеками.
\end{enumerate}