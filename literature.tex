\section{Обзор литературы}

\subsection{α-Синуклеин}

α-Синуклеин -- небольшой белок из семейства синуклеинов. Белки данного семейства -- α-,  β-,  γ-синуклеины -- пока были найдены только у позвоночных. α-Синуклеин широко изучается \cite{george_synucleins_2002, burre_cell_2018}.

Впервые α-синуклеин был обнаружен в электрической доле мозга электрического ската в 1988 году \textit{Torpedo californica} \cite{maroteaux_synuclein:_1988}. Белок был выделен с использованием антител к очищенным холинэргическим везикулам, была определена его последовательность у \textit{Torpedo} и крысы. Была обнаружена преснаптическая везикулярная и ядерная локализация синуклеина, откуда и название "син" (синаптические везикулы) и "нуклеин" (ядро). Однако, ядерная локализация остается под вопросом \cite{burre_cell_2018}.

Открытие α-синуклеина повлекло за собой обнаржение других белков семейства -- β-, γ-синуклеинов \cite{maroteaux_rat_1991}.

Фрагмент α-синуклеина был обнаружен в сенильных бляшках мозга пациентов с болезнью Альцгеймера в 1993. Он получил название не-Aβ-амилоидный компонент (non-Aβ-amyloid component, NAC) \cite{ueda_molecular_1993}.




\subsubsection{Структура и фолдинг α-синуклеина}


α-Синуклеин -- небольшой (140 аминокислот) растворимый белок.

Аминокислотную последовательность α-синуклеина можно разделить на домены. (1)  Консервативный N-концевой домен, который содержит 4 неточных повтора 11-мера  аминокислот -- консенсус XKTKEGVXXXX. Эти 11-меры могут образовывать альфа-спираль, схожую с аполипопротеиновой спиралью \cite{george_characterization_1995}. (2) Центральный участок (61-95) содержит NAC -- не-амилоид-β компонент, гиброфобный участок, который вовлечен в образование β-листовой структуры при аггрегации, и еще три повтора KTKEGV. (3) С-концевой домен богат отрицательно-заряженными остатками и пролином \cite{breydo_-synuclein_2012}.


В клетке α-синуклеин существует в двух равновесных состояниях -- свободном  и мембран-связанном. Растворенный в цитоплазме свободный синуклеин не имеет вторичной структуры и ведет себя как внутренне неструктурированный белок \cite{burre_properties_2013, uversky_protein-chameleon:_2003}.

α-Синуклеин может связываеться с мембранами -- искусственными липосомами, липидными каплями, мембранными рафтами. При связывании, семь 11-аминокислотных повторов в последовательности α-синуклеина образуют α-спираль \cite{davidson_stabilization_1998, bisaglia_topological_2005, jao_structure_2008}, рисунок \ref{fig:alpha}. 

α-Синуклеин может принимать β-листовую структуру. Эта конформация ассоциирована с агрегацией, образованием фибрилл и отложением в тельца Леви, рисунок \ref{fig:beta}.

\begin{figure*}[h]
	\centering
	\begin{subfigure}[t]{0.39\linewidth}
		\includegraphics[width = \textwidth]{pics/alpha_struct.png}
		\caption{-конформация}\label{fig:alpha}
	\end{subfigure}
	\begin{subfigure}[t]{0.6\textwidth}
		\includegraphics[width = \textwidth]{pics/beta_struct.png}
		\caption{конформация}\label{fig:beta}

	\end{subfigure}
	\caption{Конформации α-синуклеина.   
		(a) мембран-связанная, α-спиральная конформация, PDB ID: 1XQ8, \cite{ulmer_structure_2005};
		(b) фибриллы, аггрегированные \textbeta-листы, PDB ID: 2N0A, \cite{tuttle_solid-state_2016}.
		}
	\label{fig:structs}
	
\end{figure*}

Внутренне неструктурированные белки (известно около 100) обычно имеют амикислотную последовательность, которая препятствует агрегации и вообще формированию устойчивой конформации. Этому служит большое содержание заряженных остатков, пролина, отсутствие гирдрофобных стречей \cite{radivojac_intrinsic_2007}. 

В синуклеине такую роль выполняют N и C концевые домены, которые поддерживают белок в довольно компактном, динамическом состоянии, прикрывая NAC домен, так как он наиболее склонен в агрегации. Различные дальние (long-ranged) взаимодействия N и C доменов предотвращают агрегацию \cite{bertoncini_release_2005}.

Однако, различные мутации, пост-трансляционные модификации, изменения условий среды часто приводят к нарушению равновесия несвернутой конформации, заставляя белок фибрилизоваться.

Скорость образования фибрилл и олигомеров \textit{in vitro} линейно зависит от концентрации свободного растворенного белка, сильно ускоряется при наличии сформированных ядер нуклеации \cite{wood_-synuclein_1999}.





\subsubsection{Физиологическая роль}

Физиологическая роль α-синуклеина не ясна. Белки семейства синуклеинов были найдены только у позвоночных. 

α-Синуклеин, в основном, экспрессирован в мозге \cite{jakes_identification_1994} -- неокортексе, гиппокампе, стриатуме, таламусе, мозжечке \cite{iwai_precursor_1995}. Однако, также был найден в других тканях в сравнительно незначительных количествах. 

В отличие от других белков, ассоциированных с нейродегенерацией, синуклеин специфично локализуется в пресинаптических областях, его мало в теле нейронов, дендритах и в постсинаптических сайтах \cite{george_characterization_1995, iwai_precursor_1995}. 

Одна из ведущих концепций состоит в том, что  α-синуклеин выполняет функции в пресинаптических окончаниях нейронов и регулирует синаптическую передачу. В пользу этого говорит пресинаптическая локализация белка и его колокализация с резервным пулом пресинаптических везикул \cite{lee_-synuclein_2008}.

Тем не менее, в процессе развития синапсов синуклеин одним из последних колокализуется с синапсами, после интегральных белков синаптических пузырьков и синапсинов \cite{withers_delayed_1997}.

Аналогично α-синуклеину, β-синуклеин также локализуется пресинаптически \cite{jakes_identification_1994}. α- и  β- синуклеины действительно колокализуются, но не во всех синапсах. Однако, γ-синуклеин экспрессируется глиальными клетками и только в определенных нейрональных популяциях -- допаминэргических нейронов \cite{galvin_differential_2001}. γ-Синуклеин также экспрессируется в раковых опухолях (груди, прямой кишки, поджелудочной), где, возможно, играет роль в разрастании опухоли. 

 

Можно выделить следующие функции синуклеина:
\paragraph{Участие в липидном обмене и моделирование мембран} α-Синуклеин может служить переносчиком жирных кислот \cite{sharon_-synuclein_2001}.
Он также вызывает значительное искривление мембраны, превращая большие везикулы в трубочки и пузырьки  \cite{westphal_monomeric_2013} .
 
Биохимически α- и β-синуклеины были очищены как ингибиторы фосфолипазы D2 (PLD2) \cite{jenco_regulation_1998}. PLD2 гидролизует фосфатидилхолин до холина и фосфатидной кислоты и регулирует транспорт мембран, экзо- и эндоцитоз, организацию цитоскелета \cite{colley_phospholipase_1997}. В отличие от изоформы PLD1, которая активируется некоторыми клеточными каскадами, PLD2 обладает конститутивной активностью. Ингибиторная роль синуклеина была подтверждена в дрожжах \cite{outeiro_yeast_2003}, клетках млекопитающих \cite{ahn_-synuclein_2002}, \textit{in vitro}, причем была рассмотрена роль различных доменов белка в ингибировании, \cite{payton_structural_2004}. Однако, последние работы предполагают, что синуклеин не ингибирует PLD2 напрямую \cite{rappley_evidence_2009} и что данный эффект имеет причину гораздо более сложную, чем взаимодействие белков и может быть опосредован со эндоретикулярным стрессом.




\paragraph{Роль в везикулярном транспорте} Агрегаты синуклеина нарушают везикулярный транспорт в клетках млекопитающих \cite{gosavi_golgi_2002}. Мутации, связанные с БП, демонстрируют такой же эффект \cite{thayanidhi_alpha-synuclein_2010}. 

\paragraph{Регуляция обмена дофамина} Синуклеин подавляет экспрессию и активность тирозин редуктазы (ТР) -- фермент в пути биосинтеза дофамина \cite{yu_inhibition_2004} . Возрастное повышение экспрессии синуклеина коррелирует с понижением экспресии ТР \cite{chu_age-associated_2007} . Синуклеин нарушает обмен дофамина, ингибируя белок-переносчик дофамина в везикулах VMAT2 \cite{guo_inhibition_2008} .

\paragraph{Потенциальный шаперон} Синуклеин может иметь функции шаперона. Он структурно и функционально гомологичен 14-3-3 семейству шаперонов \cite{ostrerova_alpha-synuclein_1999}. С-концевой домен синуклеина подавляет аггрегацию термически денатурированных белков \cite{kim_structural_2002}. Синуклеин восстанавливает летальную делецию кошаперона CSPα в мышах, координируя сборку SNARE комплекса\cite{chandra_alpha-synuclein_2005} . Мыши -- тройные нокауты по синуклеинам демонстрируют ухудшение сборки SNARE-комплексов \cite{greten-harrison_-synuclein_2010}.

\paragraph{Выброс нейромедиатора и синаптичеcкая пластичность}
Синуклеин взаимодействует с синаптическими везикулами \cite{maroteaux_synuclein:_1988} и синаптобревином-2 \cite{burre_-synuclein_2010}, а также участвует в пересборке SNARE-комплексов как шаперон.
Уровень мРНК α-синуклеина (в этой работе названного синелфином "synelfin") значительно меняется в период обучения пению у певчих воробьиных \cite{george_characterization_1995}. Относительно всего мозга, где уровень синуклеина остается высоким на протяжение развития и взросления, в определенных областях, вовлеченных в пение, экспрессия синуклеина значительно и продолжительно падает, что предполагает участие белка с регулировке синаптической пластичности. 

Тем не менее α-cинуклеин нокаутные мыши живут нормально. У них наблюдается незначительное повышение выброса дофамина из нейронов вместе с понижением его содержания внутри клеток, нейроны быстрее восстанавливаются после нескольких стимуляций  \cite{abeliovich_mice_2000}. Двойные нокаутные мыши по α- и  β-синуклеинам также жизнеспособны, имеют нормальную морфологию мозга и синапсов, но уровень дофамина у них понижен \cite{chandra_double-knockout_2004}. Тройные нокауты, однако, демонстрируют меньший срок жизни, возрастные дисфункции нервной системы, изменение морфологии синапсов -- уменьшение размеров пресинаптических терминалей \cite{greten-harrison_-synuclein_2010}.

Учитывая специфичность синуклеина для позвоночных, а также позднюю локализацию около синапсов, можно предполагать, что он не является ключевым фактором их развития, хотя и влияет на функционирование в долгосрочной манере. 



\subsection{Патобиология α-синуклеина}

В 1977 было обнаружено, что α-синуклеин -- основной компонент телец и невритов Леви \cite{spillantini_-synuclein_1997} -- белковых клеточных включений, которые являются одним из основных признаков болезни Паркинсона (БП) и деменции с тельцами Леви (ДТЛ) -- что получило названии патологии Леви. 

Деменция с тельцами Леви очень похожа на идиопатические случаи БП по нейропатологии, при этом клинические проявления болезни другие. %ДТЛ характеризуется когнитивным растройством, паркинсонизмом, резкими колебаниями внимания и интеллекта и визуальными галлюцинациями.

α-Синуклеин также вовлечен в множественную системную атрофию (МСА).
В МСА α-синуклеин накапливается в глиальных клетках -- олигодендроцитах. Экспрессия синуклеина в таких культурах воспроизводит фенотип МСА, что позволяет считать синуклеин причиной патологии. В норме олигодендроциты не экспрессируют синуклеин, что поднимает вопрос -- как белок оказывается в клетках, захватывает ли глия белок из нейронов или экспрессия активируется неким патологическим процессом? При этом наследственные формы МСА очень редки \cite{bendor_function_2013}.

Синуклеин также вовлечен в другие нейродегенеративные заболевания -- первичную вегетативную недостаточность, диффузную болезнь Леви, нейродегенерацию с аккумуляцией железа типа I \cite{burre_cell_2018}. Около 60\% случаев болезни Альцгеймера сопровождаются патологией Леви, которая, однако, обычно обнаруживается только в миндалевидном теле \cite{hamilton_lewy_2006}.

\subsubsection{Болезнь Паркинсона}
Болезнь Паркинсона -- нейродегенеративное заболевание, которое оюнаруживается у 2-3\% населения старше 65 лет \cite{poewe_parkinson_2017}, в мире с этим заболеванием живут около 10 миллионов человек (данные Parkinson’s Foundation). 

Основные признаки БП -- потеря нейрональных функций в некоторых областях мозга и широкое внутриклеточное накопление α-синуклеина. Потеря допаминэргических нейронов на ранних стадиях ограничена черной субстанцией, но далее наблюдается в других частях мозга. 
Патология Леви обнаруживается в холинэргических и моноаминэргических нейронах ствола мозга и в нейронах обонятельной системы, с прогрессией заболевания -- в либмической системе и неокортексе.

Большинство случаев БП -- идиопатические с поздним началом, генетически обусловлены около 5-10\% патологий \cite{klein_genetics_2012}. 

Ду- или трипликации гена синуклеина обуславливают БП с ранним началом  и значительной деменцией \cite{chartier-harlin_alpha-synuclein_2004, ahn_-synuclein_2008, singleton_-synuclein_2003}. Некоторые полиморфизмы в регуляторных элементах гена синуклеина также ассоциированы с БП с ранним началом \cite{maraganore_collaborative_2006}.

Некоторые мутации миссенс α-синуклеина ассоциированы с классическим проявлением БП, наследуются по аутосомно-доминантному типу: A30P \cite{kruger_ala30pro_1998}, E46K \cite{zarranz_new_2004}, H50Q \cite{appel-cresswell_alpha-synuclein_2013}, G51D\cite{lesage_g51d_2013}, A53T\cite{puschmann_swedish_2009}, A53E \cite{pasanen_novel_2014} (OMIM entry 163890).

 Все они расположены в N-концевом домене белка. Мутация A30P ухудшает связывание синуклена с мембранами \cite{jo_defective_2002}, E46K -- повышает сродство к мембранам и ускоряет агрегацию \cite{choi_mutation_2004}, H50Q -- понижает растворимость \cite{porcari_h50q_2015}. Интересно заметить, что у грызунов в позиции 53 в норме находится треонин (для людей мутация A53T токсична) и БП не наблюдается.
 
 
Другие гены, ассоциированные с наследуемыми, моногенными формами БП -- богатая лейцином повторная киназа 2 (leucine-rich repeat kinase 2, LRRK2) , Е3 убиквитин лигаза Паркин (Parkin), митохондриальная PTEN-индуцируемая киназа 1 (Pink1), шаперон DJ-1, лизосомная АТФаза ATP13A2 \cite{klein_genetics_2012}.


Не ясно, как именно дегенерация нервных клеток в БП соотносится с аккумуляцией синуклеина.  
В черной субстанции потеря нейронов может наступать раньше появления моторных синдромов. Сама по себе патология Леви не всегда сопровождается потерей клеток -- ведь синуклеин накапливается по всему мозгу. Более того, накопление синуклеина может и не приводить ни к какой патологии \cite{markesbery_lewy_2009}. Возможно, синуклеин играет защитную роль, тогда как его мутантные форму приводят к дисфункции \cite{bendor_function_2013}.



\subsection{Моделирование токсичности синуклеина на дрожжах \emph{S.cerevisiae}}

Пекарские дрожжи \emph{Saccharomyces cerevisiae} не имеют ортологов синуклеина. Тем не менее, они используются как удобная модель для изучения синуклеина.

Такая простая модель имеет свои недостатки -- например, на дрожжах невозможно смоделировать клеточные взаимодействия, которые присутствуют в тканях многоклеточных, не говоря уже о нервных синапсах, в образовании и функционировании которых синуклеин может принимать участие. Тем не менее, участие синуклеина во внутриклеточных каскадах и взаимодействие с другими белками может быть широко изучено на дрожжах, которые легко поддаются генетическим манипуляциям.

Первая модель БП была основана на гетерологической экспрессии синуклеина в дрожжах \cite{outeiro_yeast_2003}.

Токсичность синуклеина в дрожжах зависит от уровня его экспрессии \cite{outeiro_yeast_2003}, что согласуется с существованием форм БП, вызванных ду- и трипликациями гена синуклеина.

Также α-синуклеин ассоциируется с плазматической мембраной, куда доставляется системой секреции \cite{dixon_-synuclein_2005}. Накопление белков, ассоциированных с мембраной, ведет к образованию цитоплазменных включений в концентрационно-зависимой манере \cite{dixon_-synuclein_2005}. 

Аналогично тельцам Леви в мозге с БП, многие включения α-синуклеина в клетках дрожжей окрашиваются тиофлавином-S, следовательно содержат амилоидные фибриллы \cite{zabrocki_phosphorylation_2008}, и тиофлавином Т \cite{oien_clearance_2009}. Было также продемонстрировано, что в клетках дрожжей образуются и олигомеры синуклеина \cite{tenreiro_phosphorylation_2014}.
При этом некоторые включения представляют собой агрегированные мембранные пузырьки \cite{soper_-synucleininduced_2008}.

Дрожжи понижают среднее число  мультикопийных плазмид с синуклеином на клетку, вероятно, для подавления его токсичности \cite{outeiro_yeast_2003}. Во избежание этого, для поддержания высокого уровня экспрессии, последовательность синуклеина интегрируется в геном в разном числе \cite{cooper_-synuclein_2006, su_compounds_2010}, либо используются сильные промоторы, например, галактозный GAL1 \cite{outeiro_yeast_2003}.

 Экспрессию синуклеина сопровождает митохондриальный стресс \cite{su_compounds_2010}. 
При этом система экспрессии под контролем промотора MET25 показала, что для проявления токсичности синуклеина необходимы функционирующие митохондрии \cite{buttner_functional_2008}.  

Оверэкспрессия синуклеина в дрожжах вызывает нарушение работы протеасомной системы \cite{outeiro_yeast_2003}, 
	накопление липидных капель \cite{outeiro_yeast_2003},
	стресс эндоплазматического ретикулума \cite{cooper_-synuclein_2006},
	активацию ответа на тепловой шок \cite{yeger-lotem_bridging_2009},
	дисфункцию митохондрий \cite{su_compounds_2010},
	укорочение срока жизни и индукцию митофагии \cite{sampaio-marques_snca_2012},
	нарушение эндоцитоза \cite{outeiro_yeast_2003},
	выделение активных форм кислорода и индукцию апоптоза \cite{su_compounds_2010, flower_ygr198w_2007}. 
	Было показано, что апоптоз вызывается транслокацией эндонуклеазы~G из митохондрий в ядро, где она осуществляет деградацию ДНК \cite{buttner_endonuclease_2013}.
	

В дрожжах найдено 4 гомологичных и высококонсервативных гена, принадлежащих суперсемейству DJ-1:  Hsp31, Hsp32, Hsp33 и Hsp34. Человеческий белок DJ-1, оверэкспрессированный в дрожжах, взаимодействует с синуклеином, а мутации в DJ-1, связанные с болезнью Паркинсона, нарушают это взаимодействие. DJ-1 и его гомологи в дрожжах снижали токсичность синуклеина, препятствуя его олигомеризации \cite{zondler_dj-1_2014}. 



\subsubsection{Скрининги}

Дрожжи являются удобным объектом для проведения генетических скринингов.

В первом проведенном скрининге использовалась коллекция из 4850 делеционных штаммов \cite{willingham_yeast_2003}. Было определено 86 генов, которые увеличивают токсичность синуклеина, из которых 29\% вовлечены в везикулярный транспорт и метаболизм липидов.
Особо отметим четыре гена -- VPS24, VPS28, VPS60 or SAC2 -- делеция которых увеличивала токсичность синуклеина. Они вовлечены в сортировку белков в поздних цистернах Гольджи и доставку в вакуоли. 

 Позже, с использованием той же коллекции и применением флуоресцентной микроскопии, было выявлено 185 модификаторов агрегации синуклеина \cite{zabrocki_phosphorylation_2008}. Выяснилось, что белки, регулирующие везикулярный транспорт, меняют локализацию синуклеина в клетке.

В оверэкспрессионной скрининге было обнаружено, что синуклеин блокирует транспорт от эндоплазматического ретикулума к аппарату Гольджи (ER-to-Golgi) \cite{cooper_-synuclein_2006}. Это привело к обнаружению белка Rab1, гомолога белка млекоптающих Ypt1, который оказался нейропротектором, который восстанавливает ER-to-Golgi транспорт в дрожжах, а его близкие гомологи восстанавливают здоровый фенотип нейронов с БП в различных животных моделях -- в нервной системе нематоды и в первичной культуре нейронов крысы с БП \cite{gitler_parkinsons_2008}.

Ypp1 -- жизненно необходимый белок дрожжей с неизвестной функцией,  как оказалось, при оверэкспресии опосредует транспорт синуклеина (мутантного, A30P) в вакуоли по эндоцитозному пути и, таким образом, восстанавливает нормальный фенотип клеток \cite{flower_ygr198w_2007}.


Оверэкспрессия 40 генов подавляет повышенную чувствительность дрожжей, экспрессирующих синуклеин, к действию пероксида водорода, что говорит о защитной функции этих генов против токсичности синуклеина \cite{liang_novel_2008}. Найденные гены вовлечены в убиквитин-зависимый катаболизм белков, везикулярный транспорт, ответ на стресс. 

Среди генов, отмеченных исследователями -- ARG2, ENT3, IDP3, JEM1 -- не жизеннно важные и имеют человеческих ортологов, показали наибольшую защитную функцию от активных форм кислорода, а HSP82 кодирует широко распространенный шаперон (Hsp90p). 
	Arg2p -- митохондриальный фермент, который катализирует первую реакцию в биосинтезе орнитина. Ent3p вовлечен в связывание клатрина и транспорт белков между транс-Гольджи и вакуолью. Idp3p -- пероксисомный вариант изоцитратдегидрогеназы. Jem1p -- шаперон, локализованный на эндоплазматическом ретикулуме, участвует в слиянии ядерных мембран при спаривании. Оказалось, что делеция данных генов действительно ухухдшает выживаемость дрожжей при экспрессии синуклеина.


Комплексное исследование взаимодействия синуклеина с различными клеточными путями, с использованием коллекции из 3500 штаммов, оверэкспрессирующих различные белки, позволило составить карту белков и генов, связанных с экпрессией синуклеина в клетке. Путь биосинтезе эргостерола и TOR каскад, как было найдено, влияют на эффекты, оказываемые синуклеином в дрожжах \cite{yeger-lotem_bridging_2009}.













































